\documentclass[compress,12pt]{beamer}

% Libraries
\usepackage{xcolor}
\usepackage{ragged2e}
\usepackage{algorithm}
\usepackage[noend]{algpseudocode}
\usepackage{subcaption}
\usepackage{graphicx}
\usepackage{amsmath,bm}
\usepackage{float}
\usepackage{animate}
\usepackage{hyperref}
\usepackage{pgfgantt}
\usepackage[font=small,labelfont=bf]{caption}

% Define color for the gantt diagram
\definecolor{Arno}{RGB}{255, 102, 102}
\definecolor{JJ}{RGB}{102, 204, 255}
\definecolor{Keryann}{RGB}{153, 255, 153}

% Themes
\usetheme{Arguelles}

% Page numbering
\setbeamertemplate{footline}{
  \usebeamercolor[fg]{page number in head}%
  \usebeamerfont{page number in head}%
  \hspace{345pt}
  \vspace{-35pt}
  \insertframenumber /\inserttotalframenumber
  \vspace{40pt}
}

% Allow optional arguments after frame.
\apptocmd{\frame}{}{\justifying}{} 

% Title page
\title{G\'en\'eration de labyrinthes PACMAN}
\subtitle{TER S5, M1 Informatique}
\date{7 Novembre 2025}
\author{Lesage Arno, Razafindrabe Keryann, Viale Jean-Jacques}
\github{KeryannR/TER\_S1\_F}
\event{}
\date{}
\institute{EUR DS4H - Universit\'e C\^ote d'Azur}

% Begin document
\begin{document}

    % Set up title page and table of content
    \frame[plain]{\titlepage}
    \frame[plain]{\tableofcontents, \frametitle{\textcolor{white}{\textbf{Sommaire}}}}

    % Set up part frame and sections
    %\begin{frame}[standout, plain]
    %    \centering\large
    %    \textbf{\scshape 1. Algorithmes de g\'en\'eration}
    %\end{frame}
    %\section{\textbf{1. Algorithmes de g\'en\'eration}}
    %
    %    % What is a PACMAN Game ?
    %    \begin{frame}[t]
    %        \frametitle{Qu'est ce qu'un labyrinthe PACMAN ?}
    %        \begin{columns}
    %            \begin{column}{0.5\textwidth}
    %                \vspace{20pt}
    %                \begin{figure}[H]
    %                    \centering
    %                    \includegraphics[width=0.7\textwidth]{logo/PACMAN.jpg}
    %                    \caption{Un labyrinthe PACMAN}
    %                    \label{fig:pacman.png}
    %                \end{figure}
    %            \end{column}
    %            \begin{column}{0.5\textwidth}
    %                \justifying
    %                Un labyrinthe PACMAN est générallement caractérisé par :
    %                \vspace{10pt}
    %                \begin{itemize}
    %                    \item La présence de boucles,
    %                    \item La présence de zones inaccessibles,
    %                    \item Une 4-connexité,
    %                    \item Une presque symmétrie.
    %                \end{itemize}
    %            \end{column}
    %        \end{columns}
    %    \end{frame}
    %
    %    \begin{frame}[t]
    %        \justifying
    %        \frametitle{Un premier algo : \textit{Hunt \& Kill}}
    %        \begin{columns}
    %            \begin{column}{0.7\textwidth}
    %                \begin{algorithm}[H]
    %                    \caption{\textit{Hunt \& Kill}}
    %                    \begin{algorithmic}[1]
    %                       \State $\forall (x,y)\in[\![1,n]\!]\times[\![1,m]\!] : c_{x,y}\gets 0$
    %                        \State Choisir une cellule $c$ aléatoirement
    %                        \While{$\exists(x,y) : c_{x,y}=0$}
    %                            \State $c_{x,y}\gets 1$
    %                            \State Choisir un voisin $c' : c'_{x,y} = 0$
    %                            \State $c \gets c'$; $c_{x,y}\gets 1$
    %                            \State Chercher $c : \exists(x,y)\ c'_{x,y}=0$
    %                        \EndWhile
    %                    \end{algorithmic}
    %                \end{algorithm}
    %                Problèmes : 
    %                \begin{itemize}[noitemsep,topsep=0pt,parsep=0pt,partopsep=0pt]
    %                    \item Labyrinthes parfaits,
    %                    \item Possible biai sur les bords,
    %                    \item Ne ressemble pas vraiment à PACMAN...
    %                \end{itemize}
    %            \end{column}
    %            \begin{column}{0.4\textwidth}
    %                \animategraphics[width=1\linewidth]{10}{logo/HuntAndKillFrame/frame_}{0001}{0112}
    %                \vspace{30pt}
    %            \end{column}
    %        \end{columns}
    %    \end{frame}
    %
    %    \begin{frame}[t]
    %        \frametitle{Une deuxième idée...}
    %        \framesubtitle{Tetris}
    %        \begin{columns}
    %            \begin{column}{0.5\textwidth}
    %                \vspace{10pt}
    %                \begin{figure}[H]
    %                    \centering
    %                    \includegraphics[width=1\textwidth]{logo/tiles.png}
    %                    \caption{Quelques Polyominos}
    %                    \label{fig:tiles.png}
    %                \end{figure}
    %            \end{column}
    %            \begin{column}{0.5\textwidth}
    %                \vspace{20pt}
    %                \begin{figure}[H]
    %                    \centering
    %                    \includegraphics[width=1\textwidth]{logo/OriginalTetris.png}
    %                    \caption{\href{https://shaunlebron.github.io/pacman-mazegen/}{Résultat ?}}
    %                    \label{fig:OriginalTetris.png}
    %                \end{figure}
    %            \end{column}
    %        \end{columns}
    %    \end{frame}
    %
    %    \begin{frame}[t]
    %        \frametitle{Modified Tetris : \textit{Algorithme}}
    %        \framesubtitle{Tetris}
    %        \begin{algorithm}[H]
    %            \caption{\textit{Modified Tetris}}
    %            \begin{algorithmic}[1]
    %                \State Soit $\mathbb{P}$ l'ensemble des polyominos disponibles
    %                \For{$i=0,\ i<n,\ (n\in\mathbb{N})$}
    %                    \State Choisi $p\in\mathbb{P}$
    %                    \State Faire $m\in\mathbb{N}$ rotations à $p$
    %                    \State Choisir une zone $(x,x',y,y')$ de forme $p$
    %                    \If{$\exists(c_{x,y})\in(x,x',y,y') : c_{x,y}=1$}
    %                        \State \textbf{continue}
    %                    \EndIf
    %                    \State Placer $p$ dans $(x,x',y,y')$
    %                \EndFor
    %                \State Placer la cage aux fantômes
    %                \State Retirer les "spikes" aux bords
    %                \State Retirer la 8-connexité
    %                \State Placer les portails
    %            \end{algorithmic}
    %        \end{algorithm}
    %    \end{frame}
    %
    %    \begin{frame}[t]
    %        \begin{figure}[H]
    %            \centering
    %            \frametitle{Modified Tetris : \textit{Résultat}}
    %            \framesubtitle{Tetris}
    %            \includegraphics[width=0.70\textwidth]{logo/ModifiedTetris.png}
    %            \caption{Résultat de l'algo \textit{Modified Tetris}}
    %            \label{fig:ModifiedTetris.png}
    %        \end{figure}
    %    \end{frame}

    % First section
    \begin{frame}[standout, plain]
        \centering\large
        \textbf{\scshape 1. API}
    \end{frame}
    \section{\textbf{1. API}}
    
    % --- Slide pour l'endpoint / ---
    \begin{frame}[t, fragile]
        \frametitle{Endpoint /}
        
        \textbf{Description :} Vérifie que l'API est en fonctionnement.
    
        \textbf{Méthode :} GET
    
        \textbf{Paramètres :} Aucun
    
        \textbf{Retour :} JSON avec un message de confirmation

        \vspace{25pt}  

        \textbf{Exemple :} 
        \begin{verbatim}
    GET /
    {"message": "Maze Generator API is running!"}
        \end{verbatim}
    \end{frame}

    
    % --- Slide pour l'endpoint /generate ---
    \begin{frame}[t, fragile]
        \frametitle{Endpoint /generate}
    
        \textbf{Description :} Génère dynamiquement un labyrinthe PACMAN selon les paramètres fournis.
    
        \textbf{Méthode :} GET
    
        \textbf{Paramètres :}
        \begin{itemize}[noitemsep]
            \item \texttt{xSize}, \texttt{ySize} : dimensions du labyrinthe (default 15)
            \item \texttt{minScore} : score minimal pour le labyrinthe généré
            \item \texttt{nPortal} : nombre de portails (default 1)
            \item \texttt{seed} : graine pour reproductibilité
            \item \texttt{nStep} : nombre d’itérations (default 20000)
            \item \texttt{maxBorderSpikeSize} : taille max des prolongements de bord
            \item \texttt{includeTile} : tuiles à inclure
        \end{itemize}

    \end{frame}


    % --- Slide pour le format de résultat de /generate ---
    \begin{frame}[t, fragile]
        \frametitle{Format du résultat de /generate}

        \justifying
        Exemple d'utilisation de la route :  
    
         \begin{semiverbatim}
    GET /generate?xSize=30&ySize=30&nStep=20000
        \end{semiverbatim}
        
        \justifying
        Le résultat renvoyé par l'API est un JSON structuré :  
        \scriptsize
        \begin{verbatim}
{
    "_id": "None",
    "grid": [[0,0,1,...],[...],...],
    "legend": {"0":"path","1":"wall","2":"phantom","3":"portal"},
    "metrics": {"Crossroads%":10.26, "Dead-Ends%":1.28, ...},
    "options": {"xSize":30,"ySize":30,"nStep":2000,...},
    "score": 2.65
}
        \end{verbatim}
    \end{frame}


    % --- Slide pour l'insertion mongodb ---
    \begin{frame}[t, fragile]
        \frametitle{Insertion dans MongoDB}
    
        \footnotesize
        Lorsqu'un labyrinthe est inséré dans MongoDB, la base lui attribue automatiquement un \verb|_id| unique.
        
        Cet \verb|_id| est récupéré avec le code suivant pour ensuite être ajouté au JSON renvoyé par l'API :
    
        \scriptsize
        \begin{verbatim}
    inserted_id = collection.insert_one(json_data).inserted_id
    json_data["_id"] = str(inserted_id)
        \end{verbatim}
    
        \centering
        \includegraphics[width=0.7\linewidth]{logo/mongodb_screenshot.png}
    \end{frame}
    


    
    % --- Slide pour l'endpoint /get ---
    \begin{frame}[t, fragile]
        \frametitle{Endpoint /get}
    
        \textbf{Description :} Récupère un ou plusieurs labyrinthes depuis la base selon des critères.
    
        \textbf{Méthode :} GET
    
        \textbf{Paramètres :} id, score, xSize, ySize, seed, nStep, nPortal, maxBorderSpikeSize, includeTile, limit
    
        \textbf{Retour :} JSON avec un ou plusieurs labyrinthes correspondant aux filtres.
    
        \textbf{Exemple :} 
        \small
        \begin{verbatim}
GET /get?xSize=30&ySize=30&limit=5
=> [{ maze1 }, { maze2 }, { maze3 }, { maze4 }, { maze5 }]
        \end{verbatim}
    \end{frame}






    % Second section
    \begin{frame}[standout, plain]
        \centering\large
        \textbf{\scshape 2. Tests}
    \end{frame}
    \section{\textbf{2. Tests}}

        % --- Introduction des tests ---
    \begin{frame}[t]
        \frametitle{Objectif des tests}
        \framesubtitle{Pourquoi tester ?}
        \justifying
        Les tests permettent de garantir la \textbf{stabilité} et la \textbf{fiabilité} de notre API de génération de labyrinthes.
        \vspace{10pt}
        \begin{itemize}[noitemsep,topsep=0pt]
            \item Vérifier que chaque \textbf{endpoint fonctionne correctement} (`/`, `/generate`, `/get`)
            \item Assurer la \textbf{cohérence des résultats} entre différentes exécutions
            \item Tester la \textbf{robustesse face aux paramètres invalides}
            \item Contrôler la \textbf{performance} de génération
            \item Automatiser ces vérifications via un pipeline \textbf{CI/CD GitHub Actions}
        \end{itemize}
    \end{frame}

    % --- Tests unitaires de l'API ---
    \begin{frame}[t, fragile]
        \frametitle{Tests unitaires de l'API}
        \framesubtitle{Fichier : \texttt{test\_mazeAPI.py}}
        \justifying
        Les tests unitaires reposent sur le module \texttt{unittest}. Ils vérifient chaque fonctionnalité critique de l'API :
        \vspace{5pt}
        \begin{itemize}[noitemsep,topsep=0pt]
            \item \textbf{test\_home} : Vérifie la disponibilité de l’API.
            \item \textbf{test\_generate\_default / fixed\_seed} : Teste la génération de labyrinthes.
            \item \textbf{test\_invalid\_params} : Vérifie la gestion d’erreurs.
            \item \textbf{test\_performance} : S’assure que la génération reste rapide.
        \end{itemize}
        \vspace{10pt}
        \textbf{Exemple :}
        \scriptsize
        \begin{verbatim}
    def test_generate_fixed_seed(self):
        r1 = self.client.get('/generate?xSize=10&ySize=10&seed=42')
        r2 = self.client.get('/generate?xSize=10&ySize=10&seed=42')
        self.assertEqual(r1.get_json()["grid"], r2.get_json()["grid"])
        \end{verbatim}
    \end{frame}

    % --- Tests d'intégration (CI/CD) ---
    \begin{frame}[t, fragile]
        \frametitle{Tests d'intégration et CI/CD}
        \framesubtitle{Fichier : \texttt{test\_cicd.py}}
        \justifying
        Ces tests assurent que l’API reste fonctionnelle après chaque modification majeure :
        \vspace{5pt}
        \begin{itemize}[noitemsep,topsep=0pt]
            \item Vérifie la \textbf{forme du labyrinthe généré}
            \item Vérifie la \textbf{récupération par \_id} dans MongoDB
            \item Garantit la \textbf{compatibilité du format JSON}
        \end{itemize}
        \vspace{10pt}
        \textbf{Exemple :}
        \scriptsize
        \begin{verbatim}
    def test_get_maze_by_id(self):
        test_id = "690d1d58b9055ebd0f144bb6"
        r = self.client.get(f'/get?id={test_id}')
        data = r.get_json()[0]
        self.assertIn("_id", data)
        self.assertIn("grid", data)
        \end{verbatim}
    \end{frame}

    % --- Intégration continue GitHub Actions ---
    \begin{frame}[t, fragile]
        \frametitle{Intégration Continue (CI/CD)}
        \framesubtitle{GitHub Actions - Fichier : \texttt{main.yml}}
        \justifying
        Un workflow CI/CD a été mis en place pour :
        \vspace{5pt}
        \begin{itemize}[noitemsep,topsep=0pt]
            \item Lancer automatiquement les tests à chaque \textbf{push / pull request}
            \item Valider le code avant intégration sur la branche \textbf{main}
        \end{itemize}
        \vspace{10pt}
        \scriptsize
        \begin{verbatim}
    name: MazeGeneration CICD
    on: [push, pull_request]
    jobs:
      test:
        runs-on: ubuntu-latest
        steps:
          - uses: actions/checkout@v4
          - uses: actions/setup-python@v5
            with:
              python-version: '3.11'
          - run: pip install -r requirements.txt
          - run: python -m unittest discover -v
        \end{verbatim}
    \end{frame}

    % --- Résultats CI/CD ---
    \begin{frame}[t]
        \frametitle{Résultats de l'intégration continue}
        %\item
        \framesubtitle{Exemples de pipeline}
        \begin{columns}
            \begin{column}{0.5\textwidth}
                \centering
                \includegraphics[width=\linewidth]{logo/CICD/github_pass.png}
                \captionof{figure}{\textbf{Tests réussis} : tous les checks sont verts}
            \end{column}
            \begin{column}{0.5\textwidth}
                \centering
                \includegraphics[width=\linewidth]{logo/CICD/github_fail.png}
                \captionof{figure}{\textbf{Échec} : un test échoue, pipeline bloqué}
            \end{column}
        \end{columns}
        \vspace{10pt}
        \centering
        \textbf{→ Les erreurs sont détectées automatiquement avant la mise en production.}
    \end{frame}



    % Third section
    \begin{frame}[standout, plain]
        \centering\large
        \textbf{\scshape 3. \'Evaluation et m\'etriques}
    \end{frame}
    \section{\textbf{3. \'Evaluation et m\'etriques}}

        % Create frame at will in section
        %% Dire pourquoi les métriques marches
        \begin{frame}[t]
            \frametitle{Génération du labyrinthe et métriques}
            \framesubtitle{Rappel}
            \justifying
            \begin{columns}
                \begin{column}{0.5\textwidth}
                    \vspace{20pt}
                    \begin{figure}[H]
                        \centering
                        \includegraphics[width=1\textwidth]{logo/AlgoModifiedTetris.png}
                        \caption{Algorithme Tetris Modifié}
                        \label{fig:AlgoModifiedTetris.png}
                    \end{figure}
                \end{column}
                \begin{column}{0.5\textwidth}
                    \textbf{Métriques calculés:}
                    \begin{itemize}[noitemsep,topsep=0pt,parsep=0pt,partopsep=0pt]
                        \item Proportion de \textbf{carrefour}
                        \item Proportion de \textbf{jonction}
                        \item Proportion de \textbf{virage}
                        \item Proportion de \textbf{ligne droite}
                        \item Proportion de \textbf{cul-de-sac}
                        \item Proportion de \textbf{murs}
                        \item Proportion de \textbf{chemin}
                    \end{itemize}
                \end{column}
            \end{columns}
            \vspace{-5pt}
            \textbf{Question :} Comment savoir si un labyrinthe est bon ?
            \vspace{-5pt}
            \begin{itemize}[leftmargin=.5in,noitemsep,topsep=0pt,parsep=0pt,partopsep=0pt]
                \item[1.] \`A la main ? $\longrightarrow$ trop contraignant...
                \item[2.] Automatiquement ? $\longrightarrow$ oui, mais comment ? 
            \end{itemize}          
        \end{frame}

        \begin{frame}[t]
            \frametitle{Mesure d'évaluation}
            \framesubtitle{Comparaisons}
            \begin{figure}
                \includegraphics[width=.24\textwidth]{logo/pacman1Ref.png}\hfill
                \includegraphics[width=.24\textwidth]{logo/pacman2Ref.png}\hfill
                \includegraphics[width=.24\textwidth]{logo/pacman3Ref.png}
                \caption{Quelques "vrai" PACMAN}
                \label{fig:somePacman}
            \end{figure}
            \begin{columns}
                \begin{column}[t]{0.5\textwidth}
                    \textbf{En moyenne :}
                    \begin{itemize}[noitemsep,topsep=0pt,parsep=0pt,partopsep=0pt]
                        \item \textbf{Carrefours} : 2.034\% ($\uparrow$)
                        \item \textbf{Jonctions} : 11.863\% ($\uparrow$)
                        \item \textbf{Virages} : 12.213\%
                        \item \textbf{Lignes droites} : 72.923\% ($\downarrow$)
                        \item \textbf{Cul de sac} : 0.967\%
                    \end{itemize}
                \end{column}
                \begin{column}[t]{0.5\textwidth}
                    \begin{itemize}[noitemsep,topsep=0pt,parsep=0pt,partopsep=0pt]
                        \item \textbf{Murs} : 50.766\% ($\downarrow$)
                        \item \textbf{Chemins} : 46.647\% ($\uparrow$)
                    \end{itemize}
                \end{column}
            \end{columns}
        \end{frame}
        
        \begin{frame}[t]
            \frametitle{Mesure d'évaluation}
            \framesubtitle{Formule}
            \justifying
            
            Soit $\bm{\mu}$ le vecteur de la moyenne des mesures sur les trois labyrinthes et $\bm{x}$ les mesures sur le labyrinthe cible.% et $\bm{w}$ des poids:
            \begin{itemize}[noitemsep,topsep=0pt,parsep=0pt,partopsep=0pt]
                \item[] $\bm{\mu}=\left(\mu_\text{cross},\mu_\text{junc},\mu_\text{turn},\mu_\text{straight},\mu_\text{dead},\mu_\text{wall},\mu_\text{path}\right)$
                \item[] $\bm{x}=\left(x_\text{cross},x_\text{junc},x_\text{turn},x_\text{straight},x_\text{dead},x_\text{wall},x_\text{path}\right)$
                %\item[] $\bm{w}=\left(w_\text{cross},w_\text{junc},w_\text{turn},w_\text{straight},w_\text{dead},w_\text{wall},w_\text{path}\right) = \bm{1}$
            \end{itemize}

            \fbox{
                \parbox{\textwidth}{
                    Nous définissons $\textbf{score}\left(\bm{\mu},\bm{x}\right)$ comme :
                    \begin{columns}
                        \begin{column}[t]{0.43\textwidth}
                            \centering
                            $\ \textbf{score}\left(\bm{\mu},\bm{x}\right) = 5\times\frac{\bm{x}\cdot\bm{\mu}}{\|\bm{x}\|\|\bm{\mu}\|}$
                        \end{column}
                        \begin{column}{0.57\textwidth}
                            \begin{itemize}[noitemsep,topsep=0pt,parsep=0pt,partopsep=0pt]
                                \item[] $\mu_\text{wall} > 0.8 \Rightarrow \textbf{score}\left(\bm{\mu},\bm{x}\right) = 0$
                                \item[] $\mu_\text{path} > 0.8 \Rightarrow \textbf{score}\left(\bm{\mu},\bm{x}\right) = 0$ 
                                %\item[] $\mu_\text{cross} > 0.15 \Rightarrow \textbf{score}\left(\bm{\mu},\bm{x}\right) = 0$
                            \end{itemize}
                        \end{column}
                    \end{columns}
                }
            }
            
            \hspace{-20pt}
            \textbf{Interprétation :}
            \begin{columns}
                \begin{column}{0.6\textwidth}
                    \begin{itemize}[noitemsep,topsep=0pt,parsep=0pt,partopsep=0pt]
                        \item[] Médiocre : $\textbf{score}\left(\bm{\mu},\bm{x}\right) \in [0,2[$
                        \item[] Très mauvais : $\textbf{score}\left(\bm{\mu},\bm{x}\right) \in [2,2.5[$
                        \item[] Mauvais : $\textbf{score}\left(\bm{\mu},\bm{x}\right) \in [2.5,3[$
                    \end{itemize}
                \end{column}
                \begin{column}{0.6\textwidth}
                    \begin{itemize}[noitemsep,topsep=0pt,parsep=0pt,partopsep=0pt]
                        \item[] Moyen : $\textbf{score}\left(\bm{\mu},\bm{x}\right) \in [3,3.5[$
                        \item[] Bien : $\textbf{score}\left(\bm{\mu},\bm{x}\right) \in [3.5,4[$
                        \item[] Très bien : $\textbf{score}\left(\bm{\mu},\bm{x}\right) \ge 4$
                    \end{itemize}
                \end{column}
            \end{columns}
        \end{frame}
        
        \begin{frame}[t]
            \frametitle{Mesure d'évaluation}
            \framesubtitle{Images}
            \begin{figure}
                \centering
                \begin{subfigure}[b]{0.24\textwidth}
                    \includegraphics[width=\textwidth]{logo/GeneratedMazes/score0maze.png}
                    \caption{Score: 0 [0]}
                \end{subfigure} \hfill
                \begin{subfigure}[b]{0.24\textwidth}
                    \includegraphics[width=\textwidth]{logo/GeneratedMazes/score1maze.png}
                    \caption{Score: 2.50 [1]}
                \end{subfigure} \hfill
                \begin{subfigure}[b]{0.24\textwidth}
                    \includegraphics[width=\textwidth]{logo/GeneratedMazes/score2maze.png}
                    \caption{Score: 2.94 [2]}
                \end{subfigure}
                \\[\smallskipamount]
                \begin{subfigure}[b]{0.24\textwidth}
                    \includegraphics[width=\textwidth]{logo/GeneratedMazes/score3maze.png}
                    \caption{Score: 3.08 [3]}
                \end{subfigure} \hfill
                \begin{subfigure}[b]{0.24\textwidth}
                    \includegraphics[width=\textwidth]{logo/GeneratedMazes/score4maze.png}
                    \caption{Score: 3.65 [4]}
                \end{subfigure} \hfill
                \begin{subfigure}[b]{0.24\textwidth}
                    \includegraphics[width=\textwidth]{logo/GeneratedMazes/score5maze.png}
                    \caption{Score: 4.05 [5]}
                \end{subfigure} \hfill
                %\caption{Quelques "vrai" PACMAN}
            \end{figure}
        \end{frame}
        

    % Fourth section
    \begin{frame}[standout, plain]
        \centering\large
        \textbf{\scshape 4. PACMAN}
    \end{frame}
    \section{\textbf{4. PACMAN}}

        % --- Introduction PACMAN ---
    \begin{frame}[t]
        \frametitle{Visualisation : PACMAN}
        \framesubtitle{But et intégration}
        \justifying
        L’objectif de cette partie est de \textbf{visualiser dynamiquement} les labyrinthes générés à travers une version jouable du jeu \textbf{PACMAN}.
        \vspace{8pt}
        Les éléments principaux affichés :
        \begin{itemize}[noitemsep,topsep=0pt]
            \item Le labyrinthe issu du JSON (\texttt{grid})
            \item Le personnage \textbf{Pac-Man} contrôlé par l’utilisateur
            \item Les \textbf{fantômes} et leur IA de déplacement
            \item Les \textbf{pellets} et \textbf{power pellets}
            \item Le \textbf{score} et les \textbf{vies restantes}
        \end{itemize}
        \vspace{10pt}
        Cette simulation est réalisée avec la bibliothèque \textbf{Pygame}.
    \end{frame}

    % --- Affichage du score et des vies ---
    \begin{frame}[t, fragile]
        \frametitle{Affichage du score et des vies}
        \framesubtitle{Fonction \texttt{draw\_score(window, score, lives)}}
        \justifying
        Le score et les vies sont affichés en haut de la fenêtre à chaque frame.
        \vspace{6pt}
        \begin{itemize}[noitemsep,topsep=0pt]
            \item \textbf{Score} : incrémenté lors de la collecte de pellets (+10) ou de power pellets (+50)
            \item \textbf{Vies} : représentées par des cercles jaunes sous le score
        \end{itemize}
        \vspace{8pt}
        \scriptsize
        \begin{verbatim}
def draw_score(window, score, lives):
    text = font_score.render(f"SCORE: {score}", True, WHITE)
    window.blit(text, (10, 10))
    for i in range(lives):
        pygame.draw.circle(window, (255,255,0), (10+i*25, 40), 8)
        \end{verbatim}
        \vspace{5pt}
        \centering
        \captionof{figure}{Affichage du score et des vies en jeu}
    \end{frame}

    % --- Power Pellets ---
    \begin{frame}[t, fragile]
        \frametitle{Les Power Pellets}
        \framesubtitle{Activation du mode "Power"}
        \justifying
        Les \textbf{Power Pellets} sont positionnés dans les coins du labyrinthe.
        Lorsqu’un joueur en mange un :
        \begin{itemize}[noitemsep,topsep=0pt]
            \item Pac-Man active le \textbf{mode Power} pendant quelques secondes
            \item Les fantômes deviennent \textbf{vulnérables}
            \item Le joueur peut les "manger" pour gagner des points supplémentaires
        \end{itemize}
        \vspace{6pt}
        \scriptsize
        \begin{verbatim}
if is_power:
    pacman.activate_power(duration=200)
    for ghost in ghosts:
        ghost.set_vulnerable(duration=200)
\end{verbatim}
        \vspace{8pt}
        \centering
        \captionof{figure}{Activation du mode Power lors d’un Power Pellet}
    \end{frame}

    % --- Déplacement fluide du Pac-Man ---
    \begin{frame}[t, fragile]
        \frametitle{Déplacement fluide de Pac-Man}
        \framesubtitle{Méthode \texttt{move()}}
        \justifying
        Le déplacement fluide est assuré par un système de direction et de contre-direction :
        \begin{itemize}[noitemsep,topsep=0pt]
            \item \texttt{next\_direction} : direction demandée par le joueur
            \item \texttt{direction} : direction actuelle, mise à jour si possible
        \end{itemize}
        \vspace{8pt}
        \scriptsize
        \begin{verbatim}
def move(self):
    new_x = self.x + self.next_direction[0]
    new_y = self.y + self.next_direction[1]
    if self.can_move(new_x, new_y):
        self.direction = self.next_direction
    self.x += self.direction[0]
    self.y += self.direction[1]
\end{verbatim}
        \vspace{6pt}
        \centering
        \captionof{figure}{Déplacement continu dans le labyrinthe}
    \end{frame}

    \begin{frame}[t]
    \frametitle{Fantômes}
    \framesubtitle{Déplacement et détection de collision}

    \justifying
    \begin{itemize}
        \item Chaque fantôme conserve :
        \begin{itemize}
            \item position actuelle \((x,y)\),
            \item position précédente \((x_{\text{prev}},y_{\text{prev}})\),
            \item direction et paramètres de sortie de la cage.
        \end{itemize}

        \item À chaque frame :
        \begin{itemize}
            \item on met à jour \((x_{\text{prev}},y_{\text{prev}}) \leftarrow (x,y)\) avant de déplacer l'entité,
            \item on calcule la nouvelle direction puis on avance d'une case.
        \end{itemize}

        \item Détection de collision avec Pac-Man :
        \begin{enumerate}[noitemsep]
            \item \textbf{Collision directe :} si \((x_{ghost},y_{ghost}) == (x_{pac},y_{pac})\).
            \item \textbf{Collision croisée (échange de cases) :} si
            \[
                (x_{ghost}^{\text{prev}},y_{ghost}^{\text{prev}}) == (x_{pac},y_{pac})
                \quad\text{et}\quad
                (x_{ghost},y_{ghost}) == (x_{pac}^{\text{prev}},y_{pac}^{\text{prev}}).
            \]
        \end{enumerate}

    \end{itemize}

\end{frame}


    % Fifth section
    \begin{frame}[standout, plain]
        \centering\large
        \textbf{\scshape 5. GitHub et collaboration}
    \end{frame}
    \section{\textbf{5. GitHub et collaboration}}

        % Create frame at will in section
        %% Mettre le comit graph
        \begin{frame}[t]
            \frametitle{Organisation}
            \framesubtitle{Gantt}
            \begin{figure}
                \includegraphics[width=\textwidth]{logo/Gantt.png}
                \caption{Diagramme de Gantt jusqu'à aujourd'hui}
            \end{figure}
        \end{frame}

        \begin{frame}[t]
            \frametitle{GitHub}
            \framesubtitle{Network Graph}
            \begin{figure}
                \includegraphics[width=\textwidth]{logo/Network Graph/Network Graph 2.png}
                \caption{Network Graph jusqu'à aujourd'hui}
            \end{figure}
            \begin{figure}
                \includegraphics[width=\textwidth]{logo/Network Graph/Commits.png}
                \caption{Commits sur le mois dernier}
            \end{figure}
        \end{frame}


        
    \End
    \begin{frame}[plain,standout]
        \centering
        \textbf{\scshape Merci de votre attention !}
        \vfill
    \end{frame}

\end{document}