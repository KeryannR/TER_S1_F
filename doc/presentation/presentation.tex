\documentclass[compress,12pt]{beamer}

% Libraries
\usepackage{xcolor}
\usepackage{ragged2e}
\usepackage{algorithm}
\usepackage[noend]{algpseudocode}
\usepackage{graphicx}
\usepackage{float}
\usepackage{animate}
\usepackage{hyperref}

% Themes
\usetheme{Arguelles}

% Page numbering
\setbeamertemplate{footline}{
  \usebeamercolor[fg]{page number in head}%
  \usebeamerfont{page number in head}%
  \hspace{345pt}
  \vspace{-35pt}
  \insertframenumber /\inserttotalframenumber
  \vspace{40pt}
}

% Allow optional arguments after frame.
\apptocmd{\frame}{}{\justifying}{} 

% Title page
\title{G\'en\'eration de labyrinthes PACMAN}
\subtitle{TER S5, M1 Informatique}
\date{7 Novembre 2025}
\author{Lesage Arno, Razafindrabe Keryann, Viale Jean-Jacques}
\github{KeryannR/TER\_S1\_F}
\event{}
\date{}
\institute{EUR DS4H - Universit\'e C\^ote d'Azur}

% Begin document
\begin{document}

    % Set up title page and table of content
    \frame[plain]{\titlepage}
    \frame[plain]{\tableofcontents, \frametitle{\textcolor{white}{\textbf{Sommaire}}}}

    % Set up part frame and sections
    %\begin{frame}[standout, plain]
    %    \centering\large
    %    \textbf{\scshape 1. Algorithmes de g\'en\'eration}
    %\end{frame}
    %\section{\textbf{1. Algorithmes de g\'en\'eration}}
    %
    %    % What is a PACMAN Game ?
    %    \begin{frame}[t]
    %        \frametitle{Qu'est ce qu'un labyrinthe PACMAN ?}
    %        \begin{columns}
    %            \begin{column}{0.5\textwidth}
    %                \vspace{20pt}
    %                \begin{figure}[H]
    %                    \centering
    %                    \includegraphics[width=0.7\textwidth]{logo/PACMAN.jpg}
    %                    \caption{Un labyrinthe PACMAN}
    %                    \label{fig:pacman.png}
    %                \end{figure}
    %            \end{column}
    %            \begin{column}{0.5\textwidth}
    %                \justifying
    %                Un labyrinthe PACMAN est générallement caractérisé par :
    %                \vspace{10pt}
    %                \begin{itemize}
    %                    \item La présence de boucles,
    %                    \item La présence de zones inaccessibles,
    %                    \item Une 4-connexité,
    %                    \item Une presque symmétrie.
    %                \end{itemize}
    %            \end{column}
    %        \end{columns}
    %    \end{frame}
    %
    %    \begin{frame}[t]
    %        \justifying
    %        \frametitle{Un premier algo : \textit{Hunt \& Kill}}
    %        \begin{columns}
    %            \begin{column}{0.7\textwidth}
    %                \begin{algorithm}[H]
    %                    \caption{\textit{Hunt \& Kill}}
    %                    \begin{algorithmic}[1]
    %                       \State $\forall (x,y)\in[\![1,n]\!]\times[\![1,m]\!] : c_{x,y}\gets 0$
    %                        \State Choisir une cellule $c$ aléatoirement
    %                        \While{$\exists(x,y) : c_{x,y}=0$}
    %                            \State $c_{x,y}\gets 1$
    %                            \State Choisir un voisin $c' : c'_{x,y} = 0$
    %                            \State $c \gets c'$; $c_{x,y}\gets 1$
    %                            \State Chercher $c : \exists(x,y)\ c'_{x,y}=0$
    %                        \EndWhile
    %                    \end{algorithmic}
    %                \end{algorithm}
    %                Problèmes : 
    %                \begin{itemize}[noitemsep,topsep=0pt,parsep=0pt,partopsep=0pt]
    %                    \item Labyrinthes parfaits,
    %                    \item Possible biai sur les bords,
    %                    \item Ne ressemble pas vraiment à PACMAN...
    %                \end{itemize}
    %            \end{column}
    %            \begin{column}{0.4\textwidth}
    %                \animategraphics[width=1\linewidth]{10}{logo/HuntAndKillFrame/frame_}{0001}{0112}
    %                \vspace{30pt}
    %            \end{column}
    %        \end{columns}
    %    \end{frame}
    %
    %    \begin{frame}[t]
    %        \frametitle{Une deuxième idée...}
    %        \framesubtitle{Tetris}
    %        \begin{columns}
    %            \begin{column}{0.5\textwidth}
    %                \vspace{10pt}
    %                \begin{figure}[H]
    %                    \centering
    %                    \includegraphics[width=1\textwidth]{logo/tiles.png}
    %                    \caption{Quelques Polyominos}
    %                    \label{fig:tiles.png}
    %                \end{figure}
    %            \end{column}
    %            \begin{column}{0.5\textwidth}
    %                \vspace{20pt}
    %                \begin{figure}[H]
    %                    \centering
    %                    \includegraphics[width=1\textwidth]{logo/OriginalTetris.png}
    %                    \caption{\href{https://shaunlebron.github.io/pacman-mazegen/}{Résultat ?}}
    %                    \label{fig:OriginalTetris.png}
    %                \end{figure}
    %            \end{column}
    %        \end{columns}
    %    \end{frame}
    %
    %    \begin{frame}[t]
    %        \frametitle{Modified Tetris : \textit{Algorithme}}
    %        \framesubtitle{Tetris}
    %        \begin{algorithm}[H]
    %            \caption{\textit{Modified Tetris}}
    %            \begin{algorithmic}[1]
    %                \State Soit $\mathbb{P}$ l'ensemble des polyominos disponibles
    %                \For{$i=0,\ i<n,\ (n\in\mathbb{N})$}
    %                    \State Choisi $p\in\mathbb{P}$
    %                    \State Faire $m\in\mathbb{N}$ rotations à $p$
    %                    \State Choisir une zone $(x,x',y,y')$ de forme $p$
    %                    \If{$\exists(c_{x,y})\in(x,x',y,y') : c_{x,y}=1$}
    %                        \State \textbf{continue}
    %                    \EndIf
    %                    \State Placer $p$ dans $(x,x',y,y')$
    %                \EndFor
    %                \State Placer la cage aux fantômes
    %                \State Retirer les "spikes" aux bords
    %                \State Retirer la 8-connexité
    %                \State Placer les portails
    %            \end{algorithmic}
    %        \end{algorithm}
    %    \end{frame}
    %
    %    \begin{frame}[t]
    %        \begin{figure}[H]
    %            \centering
    %            \frametitle{Modified Tetris : \textit{Résultat}}
    %            \framesubtitle{Tetris}
    %            \includegraphics[width=0.70\textwidth]{logo/ModifiedTetris.png}
    %            \caption{Résultat de l'algo \textit{Modified Tetris}}
    %            \label{fig:ModifiedTetris.png}
    %        \end{figure}
    %    \end{frame}



    % Second section
    \begin{frame}[standout, plain]
        \centering\large
        \textbf{\scshape 1. \'Evaluation et m\'etriques}
    \end{frame}
    \section{\textbf{1. \'Evaluation et m\'etriques}}

        % Create frame at will in section
        %% Dire pourquoi les métriques marches

        
    % Third section
    \begin{frame}[standout, plain]
        \centering\large
        \textbf{\scshape 2. API}
    \end{frame}
    \section{\textbf{2. API}}
    
         % Create frame at will in section
        \begin{frame}[t]
            \frametitle{Génération de labyrinthes via l'API}
            \justifying
            L'API permet de générer dynamiquement des labyrinthes PACMAN en utilisant des paramètres personnalisés.  
            \vspace{10pt}
            \textbf{Points clés :}
            \begin{itemize}[noitemsep,topsep=0pt,parsep=0pt,partopsep=0pt]
                \item \textbf{Endpoint :} \texttt{/generate} (GET)
                \item Paramètres :
                    \begin{itemize}[noitemsep,topsep=0pt]
                        \item \texttt{xSize}, \texttt{ySize} : dimensions du labyrinthe
                        \item \texttt{seed} : graine aléatoire pour la reproductibilité
                        \item \texttt{nStep} : nombre d'itérations pour la génération
                        \item \texttt{maxBorderSpikeSize} : taille maximale des prolongements de bord
                        \item \texttt{includeTile} : inclusion de tuiles "spéciales"
                    \end{itemize}
                \item Retour : JSON contenant le labyrinthe, les métriques et les options de génération
                %\item Intégration : possibilité de sauvegarder le labyrinthe dans la base de données MongoDB
            \end{itemize}
            \vspace{5pt}
            \textbf{Exemple :} \texttt{/generate?xSize=30\&ySize=30\&nStep=20000}
        \end{frame}



    % Fourth section
    \begin{frame}[standout, plain]
        \centering\large
        \textbf{\scshape 3. Tests}
    \end{frame}
    \section{\textbf{3. Tests}}

        % Create frame at will in section

    % Fifth section
    \begin{frame}[standout, plain]
        \centering\large
        \textbf{\scshape 4. PACMAN}
    \end{frame}
    \section{\textbf{4. PACMAN}}

        % Create frame at will in section

    % Six section
    \begin{frame}[standout, plain]
        \centering\large
        \textbf{\scshape 5. GitHub et collaboration}
    \end{frame}
    \section{\textbf{5. GitHub et collaboration}}

        % Create frame at will in section
        %% Mettre le comit graph


    \End
    \begin{frame}[plain,standout]
        \centering
        \textbf{\scshape Merci de votre attention !}
        \vfill
    \end{frame}

\end{document}